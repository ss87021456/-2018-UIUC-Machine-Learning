\documentclass[oneside,11pt,letter]{article}

% General include (DO NOT MODIFY)
\usepackage{amsmath,graphicx,cite,latexsym,color, amssymb,ifthen,verbatim}

\usepackage{listings}
\definecolor{mygreen}{rgb}{0,0.6,0}
\definecolor{mygray}{rgb}{0.5,0.5,0.5}
\definecolor{mymauve}{rgb}{0.58,0,0.82}

\lstset{ %
  backgroundcolor=\color{white},   % choose the background color
  basicstyle=\footnotesize,        % size of fonts used for the code
  breaklines=true,                 % automatic line breaking only at whitespace
  captionpos=b,                    % sets the caption-position to bottom
  commentstyle=\color{mygreen},    % comment style
  escapeinside={\%*}{*)},          % if you want to add LaTeX within your code
  keywordstyle=\color{blue},       % keyword style
  stringstyle=\color{mymauve},     % string literal style
}
\lstset{basicstyle=\small\ttfamily,breaklines=true}

%--------------- Various Style Declarations ----------------------------
\textheight 9in
\topmargin 0in
\headheight 0in
\headsep 0in
\textwidth 6.5in
\oddsidemargin 0in
\evensidemargin 0in
\footskip 0.2in
\parskip 5pt
\parindent 0pt
\topsep 2pt
\partopsep 0pt
\itemsep 0pt
\pagenumbering{arabic}

\definecolor{shade}{gray}{0.85}


\newcommand{\solpagesize}%
{\ifthenelse{\equal{\type}{solutions}}{
\textheight9in
\textwidth6.5in
\oddsidemargin0in
\evensidemargin0in
\topmargin-0.75in
\topskip0in
\footskip0.70in
\pagestyle{empty}
\parskip 5pt
\parindent 0pt
}{}}

\newcommand{\bookletskip}[1] %
{\ifthenelse{\equal{\type}{booklet}}{\vspace{#1 in}}
}

\newcommand{\bookletpage} %
{\ifthenelse{\equal{\type}{booklet}}{\newpage}{}
}

\newcommand{\inbooklet}[1]{\ifthenelse{\equal{\type}{booklet}}{{#1}}}


%%%%%%%%%%%%%%%%%%%%%%%%%%%%%%%%%%%%%%%
% Here are the new definitions of the commands \problem (for main
% text of problem), \problempart (for parts (a), (b) etc of the problem
% and \solution (for text of the solution).  The usage is as follows.
%
%    \begin{enumerate}
%    \problem{label}{main text of first problem}
%    \begin{enumerate}
%    \problempart{text of part(a) of first problem}
%
%    \solution{text of solution to part(a)}
%    \problempart{\text of part(b)}
%
%    \solution{text of solution to part (b)}
%
% ..... and so on for succeeding parts
%
%    \end{enumerate}
%
%    \problem{label}{main text of second problem}
%    \begin{enumerate}
%    \problempart{text of part(a)}
%
%    \solution{text of solution to part(a)}
%    \problempart{\text of part(b)}
%
%    \solution{text of solution to part (b)}
%    \end{enumerate}
%    ........ and so on for other problems
%    \end{enumerate}
%
% Please note that there needs to be a blank line separating
% a problempart command and the succeeding solution command;
% else the problem part and the solution are typeset as one
% paragraph when we are printing both the problem and its
% solution.  However, it is OK if a problempart follows a
% previous solution without an intervening blank line.  Some day I will
% waste some time figuring out a way around this problem

\newcommand{\problem}[2]%
{\item\label{#1}%
\ifthenelse{\(\equal{\type}{problems}\)\or\(\equal{\type}{both}\)}%
 {{\bf[#1]\\}#2}{{\bf[#1]}}}
 % The problem name always prints on the first line (in boldface
 % and inside square brackets.  The problem text prints on
 % succeeding lines if we are printing problems only, or problems
 % and solutions both

  \newcommand{\problempart}[1]%
{\item{\ifthenelse{\(\equal{\type}{problems}\)\or\(\equal{\type}{both}\)}%
 {#1}{}}}
 % The tag ((a), or (b) or (c) etc.) of the text of the part of the problem
 % prints in the margin, and is followed by the text of the problem beginning
 % on the same line if we are printing problems only or problems and
 % solutions both

 \newcommand{\solution}[1]%
{\ifthenelse{\equal{\type}{both}}{{\bf{Solution:\ }}{#1}}%
 {\ifthenelse{\(\equal{\type}{solutions}\)}%
 {#1}{}}}
 % This command does not generate a tag ((a), or (b) or (c) etc.)
 % for the text, but uses the tag generated by the previous 
 % problempart or examproblempart command.  If only the solutions 
 % are being printed, then the text
 % of the solution is printed beginning on the same line as the tag.
 % If both problems and solutions are being printed, then "Solution:"
 % is printed in boldface followed by the text of the solution.

  %%%%%%%%%%%%%%%%%%%%%%%%%%%%%%%%%%%%%%%%%
  
   \newcommand{\answer}[1]%
  {\ifthenelse{\equal{\type}{both}}{{\bf{Your answer:\ }}{#1}}%
  	{\ifthenelse{\(\equal{\type}{solutions}\)}%
  		{#1}{}}}
  % This command does not generate a tag ((a), or (b) or (c) etc.)
  % for the text, but uses the tag generated by the previous 
  % problempart or examproblempart command.  If only the solutions 
  % are being printed, then the text
  % of the solution is printed beginning on the same line as the tag.
  % If both problems and solutions are being printed, then "Solution:"
  % is printed in boldface followed by the text of the solution.
  
  %%%%%%%%%%%%%%%%%%%%%%%%%%%%%%%%%%%%%%%%%
  
  
 \newcommand{\examproblem}[2]%
{\item {\ifthenelse{\equal{\type}{solutions}}{}{{\bf [#1 points]} #2}}}
% The first argument is an integer specifying the number of points.  The
% first argument (followed by the word "points") is printed inside square
% brackets in boldface.  The second argument is the text of the problem
% itself.


\newcommand{\examproblempart}[1]%
{\item{\ifthenelse{\(\equal{\type}{problems}\)\or\(\equal{\type}{both}\)\or\(\equal{\type}{booklet}\)}%
 {#1}{}}}
 % The tag ((a), or (b) or (c) etc.) of the text of the part of the problem
 % prints in the margin, and is followed by the text of the problem beginning
 % on the same line if we are printing problems only or problems and
 % solutions both

%%%%%%%  ENTER SOME PROBLEM SET SPECIFIC STUFF HERE  %%%%



%%%%%%%%%%%%%%%%%%%%%%
%CHANGE
%.   to booklet to print the problems only
%
%    to both to print problems and solutions
%%%%%%%%%%%%%%%%%%%%%%

\newcommand{\type}{booklet}
%\newcommand{\type}{both}

% Custom adjustments (CHANGE THIS FILE FOR ADDITIONAL ADJUSTMENTS)
\newcommand{\cN}{{\cal N}}

\DeclareMathOperator*{\argmin}{\arg\!\min}
\newcommand{\norm}[1]{\left\lVert#1\right\rVert}

%************************************************************************
%                                                                       *
%            End of preamble and beginning of text.                     *
%                                                                       *
%************************************************************************

\begin{document}
%------------------------- Title Page ----------------------------------
\thispagestyle{empty}
\baselineskip2.5ex
{\bf University of Illinois}
\hfill
Spring 2018

{\Large
\begin{center}
{\sf CS\,446: Machine Learning}\\ Homework 1\\
\end{center}
}
{\large
\begin{center}
{\color{red}Due on Tuesday, January 23, 2018, 11:59 a.m. Central Time}
\end{center}
}



\ifthenelse{\equal{\type}{booklet}}{
\newcommand{\HWStudSolAA}{
%%%%%%%%%%%%%%%%%%%%%%%%%%%%%%%%%%%%
%%
%%.   YOUR SOLUTION FOR PROBLEM 1.a) BELOW THIS COMMENT
%%
%%%%%%%%%%%%%%%%%%%%%%%%%%%%%%%%%%%%
\vspace{2cm}
}
\newcommand{\HWStudSolAB}{
%%%%%%%%%%%%%%%%%%%%%%%%%%%%%%%%%%%%
%%
%%.   YOUR SOLUTION FOR PROBLEM 1.b) BELOW THIS COMMENT
%%
%%%%%%%%%%%%%%%%%%%%%%%%%%%%%%%%%%%%
\vspace{2cm}
}
\newcommand{\HWStudSolAC}{
%%%%%%%%%%%%%%%%%%%%%%%%%%%%%%%%%%%%
%%
%%.   YOUR SOLUTION FOR PROBLEM 1.c) BELOW THIS COMMENT
%%
%%%%%%%%%%%%%%%%%%%%%%%%%%%%%%%%%%%%
\vspace{2cm}
}

\newcommand{\HWStudSolAD}{
%%%%%%%%%%%%%%%%%%%%%%%%%%%%%%%%%%%%
%%
%%.   YOUR SOLUTION FOR PROBLEM 1.d) BELOW THIS COMMENT
%%
%%%%%%%%%%%%%%%%%%%%%%%%%%%%%%%%%%%%
\vspace{2cm}
}

\newcommand{\HWStudSolBA}{
%%%%%%%%%%%%%%%%%%%%%%%%%%%%%%%%%%%%
%%
%%.   YOUR SOLUTION FOR PROBLEM 2.a) BELOW THIS COMMENT
%%
%%%%%%%%%%%%%%%%%%%%%%%%%%%%%%%%%%%%
\begin{tabular}{|c|c|c|}
\hline
$x_1$ & $x_2$ & $y$\\
\hline
$-2.6$ & $6.6$ & ?\\
$1.4$ & $1.6$ & ?\\
$-2.5$ & $1.2$ & ?\\
\hline
\end{tabular}
}

\newcommand{\HWStudSolBB}{
%%%%%%%%%%%%%%%%%%%%%%%%%%%%%%%%%%%%
%%
%%.   YOUR SOLUTION FOR PROBLEM 2.b) BELOW THIS COMMENT
%%
%%%%%%%%%%%%%%%%%%%%%%%%%%%%%%%%%%%%
\begin{tabular}{|c|c|c|}
\hline
$x_1$ & $x_2$ & $y$\\
\hline
$-2.6$ & $6.6$ & ?\\
$1.4$ & $1.6$ & ?\\
$-2.5$ & $1.2$ & ?\\
\hline
\end{tabular}
}

\newcommand{\HWStudSolBC}{
%%%%%%%%%%%%%%%%%%%%%%%%%%%%%%%%%%%%
%%
%%.   YOUR SOLUTION FOR PROBLEM 2.c) BELOW THIS COMMENT
%%
%%%%%%%%%%%%%%%%%%%%%%%%%%%%%%%%%%%%
\vspace{2cm}
}

\newcommand{\HWStudSolBD}{
%%%%%%%%%%%%%%%%%%%%%%%%%%%%%%%%%%%%
%%
%%.   YOUR SOLUTION FOR PROBLEM 2.d) BELOW THIS COMMENT
%%
%%%%%%%%%%%%%%%%%%%%%%%%%%%%%%%%%%%%
\vspace{2cm}
}


 %The students have to fill this file to print the solution
}{
\input{HW1_OurSolution} %This file will not be provided to students since it contains the solution
}

\begin{enumerate}
%%%%%%%%%%%%%%%%%%%%%%%%%%%%%%%%%%%%%%

%%%%%  BEGINNING OF PROBLEMS LIST
% Problem Explanation:
% - first argument is the number of points
% - second argument is the title and the text
\examproblem{4}{Intro to Machine Learning}\\

Consider the task of classifying an image as one of a set of objects. Suppose we use a convolutional neural network to do so (you will learn what this is later in the semester).
\begin{enumerate}
	\examproblempart{For this setup, what is the data (often referred to as $x^{(i)}$)?\\}
	\bookletskip{0.2}
	
	\framebox[14.7cm][l]{
		\begin{minipage}[b]{14.7cm}
		\inbooklet{Your answer: \HWStudSolAA}
		
		\solution{\HWSolAA}
		\end{minipage}
	}
		
	\examproblempart{For this setup, what is the label (often referred to as $y^{(i)}$)?\\}
	\bookletskip{0.2}
	
	\framebox[14.7cm][l]{
		\begin{minipage}[b]{14.7cm}
		\inbooklet{Your answer: \HWStudSolAB}
		
		\solution{\HWSolAB}
		\end{minipage}
	}
		
	\examproblempart{For this setup, what is the model?\\}
	\bookletskip{0.2}
	
	\framebox[14.7cm][l]{
		\begin{minipage}[b]{14.7cm}
		\inbooklet{Your answer: \HWStudSolAC}
		
		\solution{\HWSolAC}
		\end{minipage}
	}
	
	\examproblempart{What is the distinction between inference and learning for this task?\\}
	\bookletskip{0.2}
	
	\framebox[14.7cm][l]{
		\begin{minipage}[b]{14.7cm}
		\inbooklet{Your answer: \HWStudSolAD}
		
		\solution{\HWSolAD}
		\end{minipage}
	}
\end{enumerate}


\examproblem{8}{$K$-Nearest Neighbors}\\

\textit{K-Nearest Neighbors} is an extension of the Nearest-Neighbor classification algorithm. Given a set of points with assigned labels, a new point is classified by considering the $K$ points closest to it (according to some metric) and selecting the most common label among these points. One common metric to use for KNN is the squared euclidean distance, i.e.
\begin{equation}
d(x^{(1)}, x^{(2)}) = \|x^{(1)} - x^{(2)}\|_2^2
\end{equation}
For this problem, consider the following set of points in $\mathbb{R}^2$, each of which is assigned with a label $y \in \{1, 2\}$:\\
\begin{figure}[h!]
\centering
\begin{tabular}{|c|c|c|}
\hline
$x_1$ & $x_2$ & $y$\\
\hline
$1$ & $1$ & 2\\
$0.4$ & $5.2$ & 1\\
$-2.8$ & $-1.1$ & 2\\
$3.2$ & $1.4$ & 1\\
$-1.3$ & $3.2$ & 1\\
$-3$ & $3.1$ & 2\\
\hline
\end{tabular}
\end{figure}
\begin{enumerate}
	\examproblempart{Classify each of the following points using the Nearest Neighbor rule (i.e. $K=1$) with the squared euclidean distance metric.\\}
	\bookletskip{0.2}
	
	\framebox[14.7cm][l]{
		\begin{minipage}[b]{14.7cm}
		\inbooklet{Your answer: \HWStudSolBA}
		
		\solution{\HWSolBA}
		\end{minipage}
	}
	\\
	\examproblempart{Classify each of the following points using the 3-Nearest Neighbor rule with the squared euclidean distance metric.\\}
	\bookletskip{0.2}
	
	\framebox[14.7cm][l]{
		\begin{minipage}[b]{14.7cm}
		\inbooklet{Your answer: \HWStudSolBB}
		
		\solution{\HWSolBB}
		\end{minipage}
	}
	\\
	\examproblempart{Given a dataset containing $n$ points, what is the outcome of classifying any additional point using the $n$-Nearest Neighbors algorithm?\\}
	\bookletskip{0.2}
	
	\framebox[14.7cm][l]{
		\begin{minipage}[b]{14.7cm}
		\inbooklet{Your answer: \HWStudSolBC	}
		
		\solution{\HWSolBC}
		\end{minipage}
	}
	\\
	\examproblempart{How many parameters are \textit{learned} when applying $K$-nearest neighbors?\\}
	\bookletskip{0.2}
	
	\framebox[14.7cm][l]{
		\begin{minipage}[b]{14.7cm}
		\inbooklet{Your answer: \HWStudSolBD}
		
		\solution{\HWSolBD}
		\end{minipage}
	}
\end{enumerate}

%%%%%%%%%%%%%%%%%%%%%%%%%%%%%%%%%%%%%%
%%%%%  BEGINNING OF SUBPROBLEMS LIST

 

 %%%%%%%%%%%% END OF SUBPROBLEMS LIST
\bookletpage
%%%%%%%%%%%% END OF PROBLEMS LIST
\end{enumerate}
\end{document}