\documentclass[oneside,11pt,letter]{article}
% General include (DO NOT MODIFY)
\usepackage{amsmath,graphicx,cite,latexsym,color, amssymb,ifthen,verbatim}

\usepackage{listings}
\definecolor{mygreen}{rgb}{0,0.6,0}
\definecolor{mygray}{rgb}{0.5,0.5,0.5}
\definecolor{mymauve}{rgb}{0.58,0,0.82}

\lstset{ %
  backgroundcolor=\color{white},   % choose the background color
  basicstyle=\footnotesize,        % size of fonts used for the code
  breaklines=true,                 % automatic line breaking only at whitespace
  captionpos=b,                    % sets the caption-position to bottom
  commentstyle=\color{mygreen},    % comment style
  escapeinside={\%*}{*)},          % if you want to add LaTeX within your code
  keywordstyle=\color{blue},       % keyword style
  stringstyle=\color{mymauve},     % string literal style
}
\lstset{basicstyle=\small\ttfamily,breaklines=true}

%--------------- Various Style Declarations ----------------------------
\textheight 9in
\topmargin 0in
\headheight 0in
\headsep 0in
\textwidth 6.5in
\oddsidemargin 0in
\evensidemargin 0in
\footskip 0.2in
\parskip 5pt
\parindent 0pt
\topsep 2pt
\partopsep 0pt
\itemsep 0pt
\pagenumbering{arabic}

\definecolor{shade}{gray}{0.85}


\newcommand{\solpagesize}%
{\ifthenelse{\equal{\type}{solutions}}{
\textheight9in
\textwidth6.5in
\oddsidemargin0in
\evensidemargin0in
\topmargin-0.75in
\topskip0in
\footskip0.70in
\pagestyle{empty}
\parskip 5pt
\parindent 0pt
}{}}

\newcommand{\bookletskip}[1] %
{\ifthenelse{\equal{\type}{booklet}}{\vspace{#1 in}}
}

\newcommand{\bookletpage} %
{\ifthenelse{\equal{\type}{booklet}}{\newpage}{}
}

\newcommand{\inbooklet}[1]{\ifthenelse{\equal{\type}{booklet}}{{#1}}}


%%%%%%%%%%%%%%%%%%%%%%%%%%%%%%%%%%%%%%%
% Here are the new definitions of the commands \problem (for main
% text of problem), \problempart (for parts (a), (b) etc of the problem
% and \solution (for text of the solution).  The usage is as follows.
%
%    \begin{enumerate}
%    \problem{label}{main text of first problem}
%    \begin{enumerate}
%    \problempart{text of part(a) of first problem}
%
%    \solution{text of solution to part(a)}
%    \problempart{\text of part(b)}
%
%    \solution{text of solution to part (b)}
%
% ..... and so on for succeeding parts
%
%    \end{enumerate}
%
%    \problem{label}{main text of second problem}
%    \begin{enumerate}
%    \problempart{text of part(a)}
%
%    \solution{text of solution to part(a)}
%    \problempart{\text of part(b)}
%
%    \solution{text of solution to part (b)}
%    \end{enumerate}
%    ........ and so on for other problems
%    \end{enumerate}
%
% Please note that there needs to be a blank line separating
% a problempart command and the succeeding solution command;
% else the problem part and the solution are typeset as one
% paragraph when we are printing both the problem and its
% solution.  However, it is OK if a problempart follows a
% previous solution without an intervening blank line.  Some day I will
% waste some time figuring out a way around this problem

\newcommand{\problem}[2]%
{\item\label{#1}%
\ifthenelse{\(\equal{\type}{problems}\)\or\(\equal{\type}{both}\)}%
 {{\bf[#1]\\}#2}{{\bf[#1]}}}
 % The problem name always prints on the first line (in boldface
 % and inside square brackets.  The problem text prints on
 % succeeding lines if we are printing problems only, or problems
 % and solutions both

  \newcommand{\problempart}[1]%
{\item{\ifthenelse{\(\equal{\type}{problems}\)\or\(\equal{\type}{both}\)}%
 {#1}{}}}
 % The tag ((a), or (b) or (c) etc.) of the text of the part of the problem
 % prints in the margin, and is followed by the text of the problem beginning
 % on the same line if we are printing problems only or problems and
 % solutions both

 \newcommand{\solution}[1]%
{\ifthenelse{\equal{\type}{both}}{{\bf{Solution:\ }}{#1}}%
 {\ifthenelse{\(\equal{\type}{solutions}\)}%
 {#1}{}}}
 % This command does not generate a tag ((a), or (b) or (c) etc.)
 % for the text, but uses the tag generated by the previous 
 % problempart or examproblempart command.  If only the solutions 
 % are being printed, then the text
 % of the solution is printed beginning on the same line as the tag.
 % If both problems and solutions are being printed, then "Solution:"
 % is printed in boldface followed by the text of the solution.

  %%%%%%%%%%%%%%%%%%%%%%%%%%%%%%%%%%%%%%%%%
  
   \newcommand{\answer}[1]%
  {\ifthenelse{\equal{\type}{both}}{{\bf{Your answer:\ }}{#1}}%
  	{\ifthenelse{\(\equal{\type}{solutions}\)}%
  		{#1}{}}}
  % This command does not generate a tag ((a), or (b) or (c) etc.)
  % for the text, but uses the tag generated by the previous 
  % problempart or examproblempart command.  If only the solutions 
  % are being printed, then the text
  % of the solution is printed beginning on the same line as the tag.
  % If both problems and solutions are being printed, then "Solution:"
  % is printed in boldface followed by the text of the solution.
  
  %%%%%%%%%%%%%%%%%%%%%%%%%%%%%%%%%%%%%%%%%
  
  
 \newcommand{\examproblem}[2]%
{\item {\ifthenelse{\equal{\type}{solutions}}{}{{\bf [#1 points]} #2}}}
% The first argument is an integer specifying the number of points.  The
% first argument (followed by the word "points") is printed inside square
% brackets in boldface.  The second argument is the text of the problem
% itself.


\newcommand{\examproblempart}[1]%
{\item{\ifthenelse{\(\equal{\type}{problems}\)\or\(\equal{\type}{both}\)\or\(\equal{\type}{booklet}\)}%
 {#1}{}}}
 % The tag ((a), or (b) or (c) etc.) of the text of the part of the problem
 % prints in the margin, and is followed by the text of the problem beginning
 % on the same line if we are printing problems only or problems and
 % solutions both

%%%%%%%  ENTER SOME PROBLEM SET SPECIFIC STUFF HERE  %%%%



%%%%%%%%%%%%%%%%%%%%%%
%CHANGE
%.   to booklet to print the problems only
%
%    to both to print problems and solutions
%%%%%%%%%%%%%%%%%%%%%%

\newcommand{\type}{booklet}
%\newcommand{\type}{both}

% Custom adjustments (CHANGE THIS FILE FOR ADDITIONAL ADJUSTMENTS)
\newcommand{\cN}{{\cal N}}

\DeclareMathOperator*{\argmin}{\arg\!\min}
\newcommand{\norm}[1]{\left\lVert#1\right\rVert}

%************************************************************************
%                                                                       *
%            End of preamble and beginning of text.                     *
%                                                                       *
%************************************************************************

\begin{document}
	%------------------------- Title Page ----------------------------------
	\thispagestyle{empty}
	\baselineskip2.5ex
	{\bf University of Illinois}
	\hfill
	Spring 2018
	
	{\Large
		\begin{center}
			{\sf CS\,446: Machine Learning}\\ Homework 5\\
		\end{center}
	}
	{\large
		\begin{center}
			{\color{red}Due on Tuesday, February 20, 2018, 11:59 a.m. Central Time}
		\end{center}
	}
	
	\ifthenelse{\equal{\type}{booklet}}{}{}
	
	\ifthenelse{\equal{\type}{booklet}}{
		\newcommand{\studSolAA}{
	%%%%%%%%%%%%%%%%%%%%%%%%%%%%%%%%%%%%
	%%
	%%.   YOUR SOLUTION FOR PROBLEM A BELOW THIS COMMENT
	%%
	%%%%%%%%%%%%%%%%%%%%%%%%%%%%%%%%%%%%
	\vspace{3cm}
}

\newcommand{\studSolAB}{
	%%%%%%%%%%%%%%%%%%%%%%%%%%%%%%%%%%%%
	%%
	%%.   YOUR SOLUTION FOR PROBLEM A BELOW THIS COMMENT
	%%
	%%%%%%%%%%%%%%%%%%%%%%%%%%%%%%%%%%%%
	\vspace{3cm}
}

\newcommand{\studSolAC}{
	%%%%%%%%%%%%%%%%%%%%%%%%%%%%%%%%%%%%
	%%
	%%.   YOUR SOLUTION FOR PROBLEM A BELOW THIS COMMENT
	%%
	%%%%%%%%%%%%%%%%%%%%%%%%%%%%%%%%%%%%
	\vspace{3cm}
}

\newcommand{\studSolBA}{
	%%%%%%%%%%%%%%%%%%%%%%%%%%%%%%%%%%%%
	%%
	%%.   YOUR SOLUTION FOR PROBLEM A BELOW THIS COMMENT
	%%
	%%%%%%%%%%%%%%%%%%%%%%%%%%%%%%%%%%%%
	\vspace{3cm}
}

\newcommand{\studSolBB}{
	%%%%%%%%%%%%%%%%%%%%%%%%%%%%%%%%%%%%
	%%
	%%.   YOUR SOLUTION FOR PROBLEM A BELOW THIS COMMENT
	%%
	%%%%%%%%%%%%%%%%%%%%%%%%%%%%%%%%%%%%
	\vspace{3cm}
}

\newcommand{\studSolBC}{
	%%%%%%%%%%%%%%%%%%%%%%%%%%%%%%%%%%%%
	%%
	%%.   YOUR SOLUTION FOR PROBLEM A BELOW THIS COMMENT
	%%
	%%%%%%%%%%%%%%%%%%%%%%%%%%%%%%%%%%%%
	\vspace{3cm}
}

\newcommand{\studSolBD}{
	%%%%%%%%%%%%%%%%%%%%%%%%%%%%%%%%%%%%
	%%
	%%.   YOUR SOLUTION FOR PROBLEM A BELOW THIS COMMENT
	%%
	%%%%%%%%%%%%%%%%%%%%%%%%%%%%%%%%%%%%
	\vspace{3cm}
} %The students have to fill this file to print the solution
	}{
		\input{multiclass_OurSolution} %This file will not be provided to students since it contains the solution
	}
	
	\begin{enumerate}
		
		%%%%%%%%%%%%%%%%%%%%%%%%%%%%%%%%%%%%%%
		%%%%%  BEGINNING OF PROBLEMS LIST
		
		\examproblem{6}{Multiclass Classification Basics}
		
		%%%%%%%%%%%%%%%%%%%%%%%%%%%%%%%%%%%%%%
		%%%%%  BEGINNING OF SUBPROBLEMS LIST
		\begin{enumerate}
			
			\examproblempart{Which of the following is the most suitable application for multiclass classification? Which is the most suitable application for binary classification?
				\begin{enumerate}
					\item Predicting tomorrow's stock price;
					\item Recognizing flower species from photos;
					\item Deciding credit card approval for a bank;
					\item Assigning captions to pictures.	
				\end{enumerate}
			}
			
			% Solution box
			\framebox[14.7cm][l]{
				\begin{minipage}[b]{14.3cm}
					
					\inbooklet{Your answer: \studSolAA}
					
					\solution{\solAA}
					
				\end{minipage}
				
			}
			
			\examproblempart{Suppose in an $n$-dimensional Euclidean space where $n\geq 3$, we have $n$ samples $x^{(i)}=e_i$ for $i=1...n$ (which means $x^{(1)}=(1,0,...,0)_n, x^{(2)}=(0,1,...,0)_n, ..., x^{(n)}=(0,0,...,1)_n$), with $x^{(i)}$ having class $i$. What are the numbers of binary SVM classifiers we need to train, to get 1-vs-all and 1-vs-1 multiclass classifiers?}
			
			% Solution box
			\framebox[14.7cm][l]{
				\begin{minipage}[b]{14.3cm}
					
					\inbooklet{Your answer: \studSolAB}
					
					\solution{\solAB}
					
				\end{minipage}
				
			}
			
			\examproblempart{Suppose we have trained a 1-vs-1 multiclass classifier from non-regularized binary SVM classifiers on the samples of the previous question. What are the regions in the Euclidean space that will receive the same number of majority votes from more than one classes?
				\textbf{Note:} You can assume for the binary SVM always outputs the \textit{perpendicular bisector} of $AB$ as decision boundary, when trained on two samples $A$ and $B$. You can ignore samples on the decision boundary of any binary SVM. }
			
			% Solution box
			\framebox[14.7cm][l]{
				\begin{minipage}[b]{14.3cm}
					
					\inbooklet{Your answer: \studSolAC}
					
					\solution{\solAC}
					
				\end{minipage}
				
			}
			
			%%%%%%%%%%%% END OF SUBPROBLEMS LIST
			
		\end{enumerate}
		
		\bookletpage
		\examproblem{8}{Multiclass SVM}
		
		%%%%%%%%%%%%%%%%%%%%%%%%%%%%%%%%%%%%%%
		%%%%%  BEGINNING OF SUBPROBLEMS LIST
		\begin{enumerate}
			Consider the objective function of multiclass SVM as
			$$\min_{w,\xi^{(i)}\geq 0}\frac{C}{2}\|w\|^2 + \sum_{i=1}^n\xi^{(i)}$$ $$ \text{s.t.} \quad w_{y^{(i)}}\phi(x^{(i)}) - w_{\hat{y}}\phi(x^{(i)}) \geq 1 - \xi^{(i)} \quad\forall i=1...n, \hat{y}=0...K-1,\hat{y}\neq y_i$$
			Let $n=K=3$, $d=2$, $x^{(1)} = (0, -1)$, $x^{(2)} = (1, 0), x^{(3)} = (0, 1), y^{(1)} = 0, y^{(2)} = 1, y^{(3)} = 2$, and $\phi(x)=x$.
			\examproblempart{
				Rewrite the objective function with $w$ being a $Kd$-dimensional vector $(w_1,w_2,w_3,w_4,w_5,w_6)^\top$ and with the specific choices of $x$, $y$ and $\phi$, following the definition of slide No.23 of Lecture 7.
			}
			
			% Solution box
			\framebox[14.7cm][l]{
				\begin{minipage}[b]{14.3cm}
					
					\inbooklet{Your answer: \studSolBA}
					
					\solution{\solBA}
					
				\end{minipage}
				
			}
			
			\examproblempart{
				Rewrite the objective function you get in (a) such that there are no slack variables $\xi^{(i)}$.
			}
			
			\framebox[14.7cm][l]{
				\begin{minipage}[b]{14.3cm}
					
					\inbooklet{Your answer: \studSolBB}
					
					\solution{\solBB}
					
				\end{minipage}
				
			}
			
			\examproblempart{
				Let $w_t=(1,1,1,2,1,-1)^\top$. Compute the derivative of the objective function you get in (b) w.r.t. $w_2$, at $w_t$, where $w_2$ is the weight of second dimension on Class 0 (in case you used non-conventional definition of $w$ in (a)).
			}
			
			\framebox[14.7cm][l]{
				\begin{minipage}[b]{14.3cm}
					
					\inbooklet{Your answer: \studSolBC}
					
					\solution{\solBC}
					
				\end{minipage}
				
			}
			
			\examproblempart{
				Prove that
				$$\max_{\hat{y}}\left(1+w_{\hat{y}}^\top\phi(x)\right)=\lim_{\epsilon\rightarrow0}\epsilon\ln\sum_{\hat{y}}\exp\left(\frac{1 + w_{\hat{y}}^\top\phi(x)}{\epsilon}\right).$$
			}
			
			\framebox[14.7cm][l]{
				\begin{minipage}[b]{14.3cm}
					
					\inbooklet{Your answer: \studSolBD}
					
					\solution{\solBD}
					
				\end{minipage}
				
			}
			
			%%%%%%%%%%%% END OF SUBPROBLEMS LIST
			
		\end{enumerate}	
		
	\end{enumerate}
\end{document}

\grid
\grid
\grid
\grid
